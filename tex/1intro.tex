\section{Introduction}

\par The N-Puzzle Problem is a sliding puzzle that consists of N moving square tiles in arranged to make make up a larger puzzle that has equal rows and columns of those tiles. In the puzzle, a tile that along with all other tiles complete all the spaces in the rows and columns is deliberately removed so that all other tiles can be rearranged or disarranged by moving adjacent tiles to the space of the supposed tile. Each tile is labeled by its original position in the grid (top to bottom, left to right). To solve the puzzle, the tiles should be moved one by one such that the original position of each tile correspond with it current position in the grid (top to bottom, left to right).\cite{bnb}

\par The puzzle was invented by a postmaster in New York as early as 1874. Since then, the puzzle has become popular by the spread of copies thus, making its way to the manufacturers and sellers. Since its invention, the puzzle started with labels using numbers from 1 to N where 1 is the position at the top left corner and N is at the bottom right (top to bottom, left to right). With the position labels, it was relatively easy to find where a out of place tile should go. Further improvement of the puzzle was done by utilizing parts of images such that when arranged correctly would result to a complete cohesive image. This way increases the difficulty of the puzzle depending on how recognizable each image part of every tile that could connect to other image parts to form the bigger image.\cite{hist}
